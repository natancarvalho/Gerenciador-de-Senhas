\documentclass[
% -- opções da classe memoir --
article,			% indica que é um artigo acadêmico
11pt,				% tamanho da fonte
oneside,			% para impressão apenas no recto. Oposto a twoside
a4paper,			% tamanho do papel. 
% -- opções da classe abntex2 --
%chapter=TITLE,		% títulos de capítulos convertidos em letras maiúsculas
%section=TITLE,		% títulos de seções convertidos em letras maiúsculas
%subsection=TITLE,	% títulos de subseções convertidos em letras maiúsculas
%subsubsection=TITLE % títulos de subsubseções convertidos em letras maiúsculas
% -- opções do pacote babel --
english,			% idioma adicional para hifenização
brazil,				% o último idioma é o principal do documento
sumario=tradicional
]{abntex2}

\usepackage[brazilian,hyperpageref]{backref}
\usepackage[alf]{abntex2cite}
\usepackage{nomencl}
\usepackage[utf8]{inputenc}	
\usepackage{indentfirst}

%opening
\title{\textbf{Gerenciador de Senhas}}
\author{Brenno Rodrigues de Carvalho da Silva\\ Natã dos Santos Carvalho}


\begin{document}
\selectlanguage{brazil}
\frenchspacing 
\maketitle

\section{Introdução}
Desenvolvido em C++ e Perl, o trabalho visa gerenciar senhas e 
verificar a qualidade da mesma, diferenciando-as entre fracas, médias 
ou fortes.

\section{Implementação}
O programa a ser desenvolvido contará com orientação a objetos. Um programa gerenciador escrito em C++ que chamará funções em Perl que fará o gerenciamento e a validação das senhas.

Dentre as definições do escopo do programa haverão cinco funções de 
interação com usuários e outras cinco de processamento de arquivo texto. 
Tal processamento se dará de forma que o programa gerenciador
em C++ permitirá ao usuário cadastrar seus e-mails e respectivas senhas. 
Ao cadastrar essas informações serão inseridas também
nome completo e data de nascimento do usuário a fim de que seja utilizado na análise das senhas.
 
O programa de interface com o usuário criará então dois arquivos textos. 
O primeiro servirá como dicionário de consultas para verificação e o segundo com as senhas que serão analisadas Só então o programa de processamento de arquivo retornará a classificação das senhas. A analise se dará de forma a verificar se as senhas possuem caracteres especiais, maiúsculos, minúsculos e dígitos  numéricos. Será verificado também se a senha se trata de um anagrama do nome do usuário trocando letras por caracteres especiais ou se contem sua data de nascimento. 

O programa gerenciador por sua vez permitirá adicionar um e-mail, senha nova, trocar senha, gerar senha aleatória e por último excluir e-mail. O mesmo será implementado em linux. 

\section{Casos de Uso}
O programa tratará com leitura do teclado, adquirindo por fim 
dados informados no tópico acima. 

\section{Conclusão}
O programa entrará em desenvolvimento após a entrega desse relatório e com prazo de entrega do processamento de arquivos textos para 12/05/2017. Após a conclusão desta etapa, será dado início ao desenvolvimento do programa gerenciador em C++.
\end{document}
