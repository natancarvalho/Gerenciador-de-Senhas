\documentclass[
% -- opções da classe memoir --
article,			% indica que é um artigo acadêmico
11pt,				% tamanho da fonte
oneside,			% para impressão apenas no recto. Oposto a twoside
a4paper,			% tamanho do papel. 
% -- opções da classe abntex2 --
%chapter=TITLE,		% títulos de capítulos convertidos em letras maiúsculas
%section=TITLE,		% títulos de seções convertidos em letras maiúsculas
%subsection=TITLE,	% títulos de subseções convertidos em letras maiúsculas
%subsubsection=TITLE % títulos de subsubseções convertidos em letras maiúsculas
% -- opções do pacote babel --
english,			% idioma adicional para hifenização
brazil,				% o último idioma é o principal do documento
sumario=tradicional
]{abntex2}

\usepackage[brazilian,hyperpageref]{backref}
\usepackage[alf]{abntex2cite}
\usepackage{nomencl}
\usepackage[utf8]{inputenc}	
\usepackage{indentfirst}
% Pacote para a definição de novas cores
\usepackage{xcolor}
% Definindo novas cores
\definecolor{verde}{rgb}{0.25,0.5,0.35}
\definecolor{jpurple}{rgb}{0.5,0,0.35}
\definecolor{darkgreen}{rgb}{0.0, 0.2, 0.13}
%\definecolor{oldmauve}{rgb}{0.4, 0.19, 0.28}
% Configurando layout para mostrar codigos Java
\usepackage{listings}

\newcommand{\estiloJava}{
	\lstset{
		language=Perl,
		basicstyle=\ttfamily\small,
		keywordstyle=\color{jpurple}\bfseries,
		stringstyle=\color{red},
		commentstyle=\color{verde},
		morecomment=[s][\color{blue}]{/**}{*/},
		extendedchars=true,
		showspaces=false,
		showstringspaces=false,
		numbers=left,
		numberstyle=\tiny,
		breaklines=true,
		backgroundcolor=\color{cyan!10},
		breakautoindent=true,
		captionpos=b,
		xleftmargin=0pt,
		tabsize=2
}}

%opening
\title{\textbf{Gerenciador de Senhas}}
\author{Brenno Rodrigues de Carvalho da Silva\\ Natã dos Santos Carvalho}


\begin{document}
\selectlanguage{brazil}
\frenchspacing 
\maketitle

\section{Introdução}
Nesta etapa do trabalho consiste na implementação das funções de leitura de arquivo, tratamento e classificação das senhas, em \verb!Perl! utilizando módulos e subrotinas.

\section{Implementação}
As funções de tratamento e classificação das senhas são executadas recebendo parâmetros como nome, data de nascimento e senha do usuário. Para a classificação das senhas foram adotados os seguintes critérios:

\begin{itemize}
	\item Utilização de caracteres numéricos - A senha recebe um 1 ponto pela utilização deste critério, caso contrário não recebe nada;
	\item Utilização de caracteres alfabéticos minúsculos - A senha recebe um 1 ponto pela utilização deste critério, caso contrário não recebe nada;
	\item Utilização de caracteres alfabéticos maiúsculos - A senha recebe um 1 ponto pela utilização deste critério, caso contrário não recebe nada;
	\item Utilização de caracteres especiais - A senha recebe um 2 ponto pela utilização deste critério, caso contrário não recebe nada;
\end{itemize}

Se o somatório dos pontos recebidos nestes 4 critérios for um total de 5 pontos, a senha é classificada como forte. Para um total de 2 ou menos pontos a senha é classificada como fraca e é recomendado a utilização dos critérios acima. 

Quando o total de pontos for igual a 3 ou 4 pontos, o programa entra em uma subrotina que verifica a ocorrência do nome o da data de nascimento na composição da senha, se houver pelo menos uma ocorrência destas, a senha é classificada como fraca e é recomendado retirar o nome e/ou a data de nascimento da mesma. 

Caso ambas as verificações citadas anteriormente sejam falsas, a senha é classificada como média e deixamos a critério do usuário melhora-la ou não.
Foram implementadas as seguintes sub rotinas:

\subsection{NumericCharacters}
Esta subrotina que detecta a incidência de um ou mais caracteres numéricos.
Caso haja este tipo de caracteres, a variável \emph{\$classificacao} recebe 1 pontos. Caso contrário, não recebe nenhuma pontuação.
\begin{scriptsize}
	\estiloJava
	\begin{lstlisting}
	sub NumericCharacters{
	  my $classificacao;
	  my @senha = @_;
	  if($senha [0] =~ /\d/){
	    #print "Tem numeros\n";
	    $classificacao = 1;
	    return $classificacao;
	}
	\end{lstlisting}
\end{scriptsize}

\subsection{LowerCaseCharacters}
Esta subrotina detecta a incidência de um ou mais caracteres alfabéticos minúsculos. Caso haja este tipo de caracteres, a variável \emph{\$classificacao} recebe 1 ponto. Caso contrário, não recebe nenhuma pontuação.

	\estiloJava
	\begin{lstlisting}
	sub LowerCaseCharacters{
	  my $classificacao;
	  my @senha = @_;
	  if($senha [0]  =~ /[a-z]/){
	    # print "Tem letras minusculas\n";
	    return $classificacao = 1;
	  }
	}
	\end{lstlisting}

\subsection{UpperCaseCharacters}
Esta subrotina detecta a incidência de um ou mais caracteres alfabéticos maiúsculos. Caso haja este tipo de caracteres, a variável \emph{\$classificacao} recebe 1 ponto. Caso contrário, não recebe nenhuma pontuação.

	\estiloJava
	\begin{lstlisting}
	sub upperCaseCharacters{
	  my $classificacao;
	  my @senha = @_; 
	  if($senha [0]  =~ /[A-Z]/){
	    #	print "Tem letras Maisculas\n";
	    return $classificacao = 1;
	  }
	}
	\end{lstlisting}
	
\subsection{SpecialCharacters}
Esta subrotina detecta a incidência de um ou mais caracteres especiais. Caso haja este tipo de caracteres, a variável \emph{\$classificacao} recebe 2 pontos. Caso contrário, não recebe nenhuma pontuação.
	\estiloJava
	\begin{lstlisting}
	sub specialCharacters{
	  my $classificacao;
	  my @senha = @_;
	  if($senha [0]  =~ /[!@#\$\%~^|&*_+]/){
	    #	print "Tem special\n";
	    return $classificacao = 2;   
	  }
	}
	\end{lstlisting}

\subsection{CheckBirthday}
Esta subrotina é executada somente se o somatório dos pontos concedidos nas subrotinas anteriores seja igual a 3 ou 4 pontos. Ela detecta a utilização da data de nascimento do usuário na composição da senha. Caso a senha possua a informação citada acima, e retornado para a rotina principal o valor 1 que indicará que a senha devera receber a classificação de senha fraca.

	\estiloJava
	\begin{lstlisting}
	sub checkBirthday {
	  my @nascimento = $_[1];
	  #print $nascimento [0];
	  my @senha = $_[0];
	  if (index($senha [0], $nascimento [0]) != -1){
 	    #print "Senha utliza a data de nascimento\n";
	    return 1;
	  }
	} 
	\end{lstlisting}

\subsection{CheckName}
Esta subrotina também é executada somente se o somatório dos pontos concedidos nas subrotinas anteriores seja igual a 3 ou 4 pontos. Ela detecta a utilização do nome do usuário na composição da senha. Caso a senha possua esta informação, é retornado para a rotina principal o valor 1 que indicará que a senha deverá receber a classificação de senha fraca.
	
	\estiloJava
	\begin{lstlisting}
	sub checkName{
	  my @nome = $_[1];
	  my @senha = $_[0];
	  my $bool =0;
	  if (index($senha [0], $nome [0]) != -1){
	    #print "Senha utliza o nome\n";
	    return $bool=1;
	  }
	}
	\end{lstlisting}	

\section{Casos de Uso}
Estas subrotinas agrupadas em um módulo serão utilizadas pelo programa principal implementado em \verb!C++!. No momento estamos usando um programa de testes para verificar a correta execução das mesmas.
 
 
\section{Conclusão}
Algumas subrotinas deverão ser revisadas posteriormente pois em alguns dos nossos testes elas não se comportaram da maneira que gostaríamos. Estaremos providenciando todas as melhorias necessárias para cumprir com o que foi proposto na primeira etapa do trabalho.

\end{document}
